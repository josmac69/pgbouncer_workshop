

% -----------------------------------------------
\begin{frame}[fragile]{Use Case: Dedicated PgBouncer per Tenant/Team}

% \vspace{-10px}
\begin{itemize}
\item Each tenant/team has its own PgBouncer instance
\item Isolates authentication and workloads between tenants/teams
\item Simplifies setting limits per tenant
\item Each tenant/team has its own pgbouncer.ini and userlist.txt files
\item Allows usage of specific per tenant/team users for database connections
\item Prevents some types of noisy neighbor issues between tenants/teams
\item Simplifies user management - one huge pg_hba.conf not needed

\bigskip
\fontsize{7}{7}\selectfont
\item \href{https://engineering.adjust.com/post/pgbouncer_authentication_layer/}{\color{blue}{Introducing a PgBouncer authentication layer into our database architecture}}
\end{itemize}

\end{frame}


% -----------------------------------------------
\begin{frame}[fragile]{Use Case: One PgBouncer Proxying Multiple Databases}
% \vspace{-10px}
\begin{itemize}
\item One PgBouncer instance can proxy connections to multiple databases
\item Simplifies connection management for applications
\item Allows for shared connection pooling across databases
\item Useful for applications with many small databases with low traffic
\item Reduces resource consumption compared to multiple PgBouncer instances
\item Require user mapping per database to avoid conflicts in user names
\item postgres1 = postgres/database1, postgres2 = postgres/database2, ...

% \fontsize{7}{7}\selectfont
% \item \href{}{\color{blue}{}}
\end{itemize}
% \begin{imagedesccent}Images created by author\\using DeepDreamGenerator\\ \end{imagedesccent}

\end{frame}


% -----------------------------------------------
\begin{frame}[fragile]{Use Case: Serverless Applications}
% \vspace{-10px}
\begin{itemize}
\item Serverless applications often have unpredictable connection patterns
\item Can produce connection spikes that overwhelm PostgreSQL server
\item PgBouncer can smooth out connection spikes with pooling and queuing
\item Reduces connection overhead for short-lived serverless functions
\item Improves overall application responsiveness and stability
\item Deploy PgBouncer as a lightweight service in the database network
\item Configure serverless functions to connect via PgBouncer port

\bigskip
\fontsize{7}{7}\selectfont
\item \href{https://medium.com/@mcnc33/serverless-connection-pooling-with-aws-lambda-and-pgbouncer-on-rds-a-step-by-step-guide-f908cab6de4b}{\color{blue}{Serverless Connection Pooling with AWS Lambda and PgBouncer on RDS: A Step-by-Step Guide}}
\end{itemize}


\end{frame}


% -----------------------------------------------
\begin{frame}[fragile]{Use Case: Read/Write Splitter}
\vspace{-10px}
\begin{itemize}
\item Some applications use separate databases for reads and writes
\item Writes go to primary database, reads from replicas
\item PgBouncer does not natively support read/write splitting
\item It does not parse SQL queries to determine type
\item Split must be done at application level via user names
\item Solution can further balance reads across replicas using HAProxy
\end{itemize}

\bigskip
\hspace{1cm}\includegraphics[width=0.3\textwidth]{logos/read-write-splitter.png}

\bigskip
\fontsize{7}{7}\selectfont
\hspace{1cm}\href{https://www.enterprisedb.com/blog/taking-advantage-write-only-and-read-only-connections-pgbouncer-django}{\color{blue}{Taking Advantage of Write-Only and Read-Only Connections in PgBouncer with Django}}

\end{frame}


% -----------------------------------------------
\begin{frame}[fragile]{Use Case: PgBouncer Listening on Multile Ports}
\vspace{-10px}
\begin{itemize}
\item Can happen after consolidation of multiple PostgreSQL servers
\item One PgBouncer instance can listen on multiple ports
\item Setting \texttt{listen\_port} allows only one port
\item But problem can be solved using systemd .socket unit
\end{itemize}

\begin{lstlisting}[frame=none, numbers=none, language=SQL,
keepspaces=true, aboveskip=10px,
basicstyle=\fontsize{8}{8}\selectfont\ttfamily]
	## /etc/systemd/system/pgbouncer.socket

	[Socket]
	ListenStream=6432
	ListenStream=6433
	ListenStream=6434
	ListenStream=/var/run/postgresql/.s.PGSQL.6432
\end{lstlisting}

% \bigskip
\fontsize{7}{7}\selectfont
\hspace{1cm}\href{https://www.percona.com/blog/configuring-pgbouncer-for-multi-port-access/}{\color{blue}{Configuring PgBouncer for Multi-Port Access}}

\end{frame}


% -----------------------------------------------
\begin{frame}[fragile]{Experimental Use Case: Pooling FDW Connections}

% \vspace{-10px}
\begin{itemize}
\item Foreign Data Wrappers (FDW) allow access to external data sources
\item Each FDW connection creates a new PostgreSQL connection
\item Can lead to many open connections and resource exhaustion
\item PgBouncer can be used to pool FDW connections
\item Configure FDW to connect via PgBouncer port
\item Reduces number of open connections to external data source
\end{itemize}

\end{frame}


% -----------------------------------------------
\begin{frame}[fragile]{Experimental Use Case: Capping User Connections}

% \vspace{-10px}
\begin{itemize}
\item PgBouncer can limit connections per user/database pair
\item Prevents a single user from exhausting all connections
\item Parameters:
\begin{itemize}
\item max\_db\_connections limits per database
\item max\_user\_connections limit per user
\item max\_user\_client\_connections limit per user over all databases
\end{itemize}
\end{itemize}

\end{frame}

% % -----------------------------------------------
% \begin{frame}[fragile]{Maintenance - Viewing Connection Pools}
% % \vspace{-10px}
% \begin{lstlisting}[frame=none, numbers=none, language=SQL,
% keepspaces=true, aboveskip=5px,
% basicstyle=\fontsize{8}{8}\selectfont\ttfamily]

% pgbouncer=# SHOW POOLS;
% -[ RECORD 1 ]---------
% database   | pgbench
% user       | pgbouncer
% cl_active  | 0
% cl_waiting | 0
% sv_active  | 0
% sv_idle    | 0
% sv_used    | 0
% sv_tested  | 0
% sv_login   | 0
% maxwait    | 0
% ...
% \end{lstlisting}
% \end{frame}


% % -----------------------------------------------
% \begin{frame}[fragile]{Maintenance - Viewing Statistics}
% % \vspace{-10px}
% \begin{lstlisting}[frame=none, numbers=none, language=SQL,
% keepspaces=true, aboveskip=5px,
% basicstyle=\fontsize{8}{8}\selectfont\ttfamily]

% pgbouncer=# SHOW STATS;
% -[ RECORD 1 ]----+-----------
% database         | pgbench
% total_requests   | 50006
% total_received   | 3163234
% total_sent       | 3200411
% total_query_time | 1821401509
% avg_req          | 0
% avg_recv         | 0
% avg_sent         | 0
% avg_query        | 0
% ...
% \end{lstlisting}
% \end{frame}


% % -----------------------------------------------
% \begin{frame}[fragile]{Maintenance - Viewing Client Connections}
% % \vspace{-10px}
% \begin{lstlisting}[frame=none, numbers=none, language=SQL,
% keepspaces=true, aboveskip=5px,
% basicstyle=\fontsize{8}{8}\selectfont\ttfamily]

% pgbouncer=# SHOW CLIENTS;
% -[ RECORD 1 ]+--------------------
% type         | C
% user         | postgres
% database     | pgbouncer
% state        | active
% addr         | unix
% port         | 6432
% local_addr   | unix
% local_port   | 6432
% connect_time | 2010-06-17 17:14:58
% request_time | 2010-06-17 17:34:14
% ptr          | 0x9ff0fc0
% link         |
% ...
% \end{lstlisting}
% \end{frame}


% % -----------------------------------------------
% \begin{frame}[fragile]{Development Improvements}
% \begin{columns}[T]
% \raggedright
% \column{0.7\textwidth}
% % \vspace{-10px}
% \begin{itemize}
% \item \textbf{Virtual Generated Columns}
% \item Columns computed on-the-fly from other columns
% \item VIRTUAL columns are default, STORED must be specified explicitly
% \item Stored column can be included in logical replication streams
% \item Virtual columns cannot be indexed or constrained

% \bigskip
% \item \textbf{Enhanced RETURNING clause}
% \item INSERT/UPDATE/DELETE/MERGE can access OLD and NEW values
% \item RETURNING old.*, new.* syntax
% \item Simplifies auditing, logging, change tracking
% \end{itemize}

% \vskip2ex
% \column{0.3\textwidth}
% \vspace{-10px}
% \begin{center}
% \includegraphics[width=\textwidth]{logos/DALL-E-2025-04-02-18-24-01-diagnose.jpg}

% % \begin{imagedesccent}Images created by author\\using DeepDreamGenerator\\ \end{imagedesccent}
% \end{center}
% \end{columns}

% \end{frame}



% % -----------------------------------------------
% \begin{frame}[fragile]{Development Improvements}
% \begin{columns}[T]
% \raggedright
% \column{0.6\textwidth}
% % \vspace{-10px}
% \begin{itemize}
% \item \textbf{Native UUIDv7 support}
% \item UUID is generally 16 bytes (128 bits) integer
% \item Just presented as hexadecimal string
% \item Function uuidv7() generates timestamp-ordered UUIDs
% \item Include 12-bit millisecond precision timestamp
% \item Replacement for random UUIDs and BIGINT IDs
% \item Improves index locality, reduces B-tree page splits
% \item But it also creates some security leakage
% \item Exposes timestamp of creating record
% \item Exposes time difference between records
% \end{itemize}

% \vskip2ex
% \column{0.4\textwidth}
% \vspace{-10px}
% \begin{center}
% \includegraphics[width=\textwidth]{logos/uuidv7.png}

% \begin{imagedesccent}Images from the article\\ \href{https://buildkite.com/resources/blog/goodbye-integers-hello-uuids/}{\color{blue}{Goodbye integers. Hello UUIDv7!}}\\ \end{imagedesccent}
% \end{center}
% \end{columns}

% \end{frame}


% % -----------------------------------------------
% \begin{frame}[fragile]{Security and Authentication Improvements}
% \begin{columns}[T]
% \raggedright
% \column{0.7\textwidth}
% % \vspace{-10px}
% \begin{itemize}
% \item \textbf{Native OAuth 2.0 authentication}
% \item Supports OAuth 2.0 token-based authentication
% \item Works with external identity providers
% \item Configured via pg\_hba.conf using method "oauth"

% \bigskip
% \item \textbf{MD5 deprecated \& SCRAM improvements}
% \item MD5 authentication officially deprecated
% \item SCRAM-SHA-256 is the recommended replacement

% \bigskip
% \item \textbf{SCRAM pass-through for postgres\_fdw and dblink}
% \item No plaintext passwords needed anymore
% \end{itemize}

% \vskip2ex
% \column{0.3\textwidth}
% \vspace{-10px}
% \begin{center}
% % \includegraphics[width=\textwidth]{logos/ales_svejk.png}

% % \begin{imagedesccent}Images created by author\\using DeepDreamGenerator\\ \end{imagedesccent}
% \end{center}
% \end{columns}

% \end{frame}


% % -----------------------------------------------
% \begin{frame}[fragile]{Constraints Improvements}
% % \begin{columns}[T]
% % \raggedright
% % \column{0.5\textwidth}
% % \vspace{-10px}
% \begin{itemize}
% \item \textbf{NOT VALID constraints}
% \item Constraints can be added as NOT VALID
% \item Existing data not checked initially
% \item Validated later using VALIDATE CONSTRAINT
% \item adding NOT NULL without downtime impact
% \end{itemize}

% % \vskip2ex
% % \column{0.5\textwidth}
% % \vspace{-10px}
% % \begin{center}
% % \includegraphics[width=\textwidth]{logos/ales_svejk.png}
% % \begin{imagedesccent}Images created by author\\using DeepDreamGenerator\\ \end{imagedesccent}
% % \end{center}

% \begin{lstlisting}[frame=none, numbers=none, language=SQL,
% keepspaces=true, aboveskip=5px,
% basicstyle=\fontsize{8}{8}\selectfont\ttfamily]

% -- Add a NOT NULL constraint without full table scan:
% ALTER TABLE orders
% ADD CONSTRAINT orders_customer_not_null
% NOT NULL (customer_id) NOT VALID;

% -- Later, validate the constraint when convenient:
% ALTER TABLE orders
% VALIDATE CONSTRAINT orders_customer_not_null;


% \end{lstlisting}
% % \end{columns}
% \end{frame}


% % -----------------------------------------------
% \begin{frame}[fragile]{Constraints Improvements}
% % \begin{columns}[T]
% % \raggedright
% % \column{0.5\textwidth}
% % \vspace{-10px}
% \begin{itemize}
% \item \textbf{Temporal constraints}
% \item WITHOUT OVERLAPS for temporal data integrity
% \item No overlapping time periods in specified columns
% \end{itemize}

% % \vskip2ex
% % \column{0.5\textwidth}
% % \vspace{-10px}
% % \begin{center}
% % \includegraphics[width=\textwidth]{logos/ales_svejk.png}
% % \begin{imagedesccent}Images created by author\\using DeepDreamGenerator\\ \end{imagedesccent}
% % \end{center}

% \begin{lstlisting}[frame=none, numbers=none, language=SQL,
% keepspaces=true, aboveskip=5px,
% basicstyle=\fontsize{8}{8}\selectfont\ttfamily]
% -- Table with a temporal primary key (id + valid period must be unique and non-
% overlapping per id):
% CREATE TABLE subscriptions (
% sub_id INT,
% valid_daterange TSRANGE,
% PRIMARY KEY (sub_id, valid_daterange WITHOUT OVERLAPS));

% -- Another table referencing the temporal key:
% CREATE TABLE usage_logs (
% sub_id INT,
% use_period TSRANGE,
% FOREIGN KEY (sub_id, PERIOD use_period)
% REFERENCES subscriptions(sub_id, PERIOD valid_daterange));
% \end{lstlisting}
% % \end{columns}
% \end{frame}


% % -----------------------------------------------
% \begin{frame}[fragile]{Constraints Improvements}
% % \begin{columns}[T]
% % \raggedright
% % \column{0.5\textwidth}
% % \vspace{-10px}
% \begin{itemize}
% \item \textbf{Unique index with NULLS DISTINCT}
% \item Allows unique indexes to treat NULLs as distinct
% \item Improves data integrity for columns that can contain NULLs
% \end{itemize}

% % \vskip2ex
% % \column{0.5\textwidth}
% % \vspace{-10px}
% % \begin{center}
% % \includegraphics[width=\textwidth]{logos/ales_svejk.png}
% % \begin{imagedesccent}Images created by author\\using DeepDreamGenerator\\ \end{imagedesccent}
% % \end{center}

% \begin{lstlisting}[frame=none, numbers=none, language=SQL,
% keepspaces=true, aboveskip=5px,
% basicstyle=\fontsize{8}{8}\selectfont\ttfamily]

% -- Before: multiple NULLs were allowed in a UNIQUE column
% CREATE TABLE users (
% email TEXT UNIQUE -- (could have multiple NULL emails)
% );
% -- Now: disallow multiple NULLs by marking NULLs "not distinct"
% CREATE TABLE accounts (
% username TEXT UNIQUE,
% email TEXT UNIQUE NULLS NOT DISTINCT -- only one NULL allowed
% );

% -- Create a unique index with NULLS NOT DISTINCT
% CREATE UNIQUE INDEX users_email_idx ON users (email)
% NULLS NOT DISTINCT;

% \end{lstlisting}
% % \end{columns}
% \end{frame}


% % -----------------------------------------------
% \begin{frame}[fragile]{Operational \& Observability Improvements}
% \begin{columns}[T]
% \raggedright
% \column{0.7\textwidth}
% % \vspace{-10px}
% \begin{itemize}
% \item \textbf{Database Checksums enabled by default}
% \item initdb enables data checksums by default
% \item Can be disabled with --no-data-checksums
% \item Improves data integrity and corruption detection

% \bigskip
% \item \textbf{Improved pg\_upgrade}
% \item Planner statistics migrated automatically
% \item ANALYZE after upgrade no longer needed
% \item New --swap method to swap data directories
% \item Parameter --jobs for parallel checks
% \item Speeds up upgrade process significantly
% \end{itemize}

% \vskip2ex
% \column{0.3\textwidth}
% \vspace{-10px}
% \begin{center}
% % \includegraphics[width=\textwidth]{logos/ales_svejk.png}

% % \begin{imagedesccent}Images created by author\\using DeepDreamGenerator\\ \end{imagedesccent}
% \end{center}
% \end{columns}

% \end{frame}


% % -----------------------------------------------
% \begin{frame}[fragile]{Backup Improvements}
% \begin{columns}[T]
% \raggedright
% \column{0.7\textwidth}
% % \vspace{-10px}
% \begin{itemize}
% \item \textbf{pg\_dumpall improvements}
% \item Can produce custom and directory format dumps
% \item Speeds up full-cluster dumps significantly

% \bigskip
% \item \textbf{fine-grained dump options}
% \item New options --with-data, --with-schema, --with-statistics
% \item And --no-data, --no-schema, --no-statistics, --statistics-only
% \item Sequence data dumps --sequence-data
% \end{itemize}

% \vskip2ex
% \column{0.3\textwidth}
% \vspace{-10px}
% \begin{center}
% % \includegraphics[width=\textwidth]{logos/ales_svejk.png}

% % \begin{imagedesccent}Images created by author\\using DeepDreamGenerator\\ \end{imagedesccent}
% \end{center}
% \end{columns}

% \end{frame}


% % -----------------------------------------------
% \begin{frame}[fragile]{Performance Improvements}
% \begin{columns}[T]
% \raggedright
% \column{0.7\textwidth}
% % \vspace{-10px}
% \begin{itemize}
% \item Speed-up the processing of queries
% \item Improved INTERSECT and EXCEPT operations
% \item Enhanced performance of window aggregates
% \end{itemize}

% \vskip2ex
% \column{0.3\textwidth}
% \vspace{-10px}
% \begin{center}
% % \includegraphics[width=\textwidth]{logos/ales_svejk.png}

% % \begin{imagedesccent}Images created by author\\using DeepDreamGenerator\\ \end{imagedesccent}
% \end{center}
% \end{columns}

% \end{frame}


% % % -----------------------------------------------
% % \begin{frame}[fragile]{Backup Improvements}
% % \begin{columns}[T]
% % \raggedright
% % \column{0.7\textwidth}
% % % \vspace{-10px}
% % \begin{itemize}

% % \end{itemize}

% % \vskip2ex
% % \column{0.3\textwidth}
% % \vspace{-10px}
% % \begin{center}
% % % \includegraphics[width=\textwidth]{logos/ales_svejk.png}

% % % \begin{imagedesccent}Images created by author\\using DeepDreamGenerator\\ \end{imagedesccent}
% % \end{center}
% % \end{columns}

% % \end{frame}


% % -----------------------------------------------
% \begin{frame}[fragile]{Resources}
% \begin{columns}[T]
% \raggedright
% \column{0.7\textwidth}
% % \vspace{-10px}
% \begin{itemize}
% \fontsize{7}{7}\selectfont
% \item \href{https://www.percona.com/blog/pgbouncer-for-postgresql-how-connection-pooling-solves-enterprise-slowdowns/}{\color{blue}{PgBouncer for PostgreSQL: How Connection Pooling Solves Enterprise Slowdowns}}
% \end{itemize}

% \vskip2ex
% \column{0.3\textwidth}
% \vspace{-10px}
% \begin{center}
% % \includegraphics[width=\textwidth]{logos/ales_svejk.png}

% % \begin{imagedesccent}Images created by author\\using DeepDreamGenerator\\ \end{imagedesccent}
% \end{center}
% \end{columns}

% \end{frame}
